\chapter{A programról}
\label{ch:about_hestia}

\section{Motiváció}

Felmerülhet a kérdés, hogy pontosan miért van szükség jelen esetben egy új programra -- git-ből kinyerhetőek a vizsgálathoz szükséges statisztikák akár csak egy egyszerű bash script segítségével, a coverage adatokat pedig a már amúgy is generált report-ból szemre meg tudjuk vizsgálni.

Egyrészt a triviális és sok szoftverfejlesztési projekt motivációjaként szolgáló válasz itt is a kényelem és automatizálás. Természetesen lehet bash script-eket használni, kézzel feldolgozni a kimenetüket, majd szemmel összenézni azt egy tetszőleges formátumú coverage report-al, de ez nem éppen ideális felhasználása egy fejlesztő idejének, főleg nem akkor, ha ezt sokszor meg kell ismételni.

Másrészt fontos megjegyezni, hogy ugyan specifikusan a vizsgálat számára releváns git statisztikák követésére léteznek megoldások (nevezetesen például a korábban már említett Adam Tornhill által fejlesztett, fizetős CodeScene, illetve az annál sokkal egyszerűbb GitNStats), ezen megoldások azonban nem használják ki a coverage által nyújtott extra lehetőségeket.

\section{Architektúra}

\section{A program használata}